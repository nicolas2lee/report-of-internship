\subsection{History of the project tracking system}
\begin{itemize}
\item{\textbf{First Step}}

The initial project was realised by Nicolas BROCHETON in 2013, which aimed to give the possibility to create a reliable tracking system to evaluate the performance of all robots during WRSC. And he decided to send all the GPS data via an Android application by the use of smartphones; also he offers a possibility to show the gathered data in a web server, but with limited amount of data.

\item{\textbf{Reborn - SWARMON}}

The SWARMON project was developed by ENSTA Bretagne's students Quentin DESCOURS, Benoit BOURDON, Jean-Jacques BOYE, Simon STEPHAN, under the guidance of Professor Olivier REYNET. Instead of using smartphones, they proposed to use the electronic cards with the special antennae, with this combination, the precision of GPS data was improved hugely. And based on the Ruby on Rails framework, the web site was built in with the essential capacities to display the gathered GPS data in the google map.

\item{\textbf{WRSC 2014}}

After the development of SWARMON,   Bastien DROUOT and Benoit BOURDON continue this project in the Åland University of Applied Sciences, under the direction of Ronny ERIKSSON, vice rector of the University during their summer internship to complete the functionalities of the tracking system. Further the developed tracking system was used successfully for the WRSC 2014 in Galway (Ireland). 

\item{\textbf{MYR}}

After the experience of WRSC 2014, Benoit BOURDON and Bastien DROUOT found the potential improvements for the tracking system, then during their third year’s project – MYR, a RESTful web site with new style was well settled. And another project in ENSTA Bretagne, which realised by Thibault VIRAVAU, provides the Android application with compatible APIs for the web site and electronic cards.
\end{itemize}