\subsection{Internship Objectives}
As all the competitive robots accomplish their missions in the sea, and it is really hard to supervise the whole traces of all robots, furthermore it is impossible to measure accurately all the positions of robots, just based on the crews' visions, especially under the extreme weather conditions. From these reasons, it is necessary to build a reliable tracking system which can gather all GPS data and time stamps in order to record all the states during certain periods not only for controlling robots' states in case of intervening collision avoidance, but also for evaluating the final performance of all the robots.

With the efforts of our predecessors, we had already defined the usable APIs for our electronic cards, which means we had already defined the APIs for gathering GPS data including latitude, longitude, date time, course and speed, but the antennae in the previous version were not suitable and specified for our desired frequency to gather GPS information, although we could also obtain all the GPS data what we want, the precision of measurement remains insufficient. So for the hardware part, we needed to choose appropriate antennae, and we needed to prepare all the materials for the tracking system before the final competition(like batteries, electronic cards, SIMCOM modules, SIM cards, and boxes which were used to put the whole tracker). Unfortunately, at the beginning of our internship, the number of participated teams was still unknown, to make sure that we got enough trackers, we booked 10 for each module. 

For software part, the previous version of website based on the web framework Ruby On Rails and the use of google map (JavaScript APIs of google map were provided and published on the Internet by Google Company freely.) And the performance of the previous version web site in Galway (Ireland, WRSC 2014) proved that the website was built successfully. However, the old CSS and HTML were obsolete and the functionality of the web site was incomplete. Additionally, the web site had already the capacity to display the GPS data gathered by the tracker on the google map, but the web site could not handle of these information further until this moment. So far, several domains remain to be improved for the web site:
\begin{itemize}
\item The security of account need to be reinforced; 
\item Replay and real-time display methods need to be improved to reduce the pressure of server;
\item Scoring and Ranking based on the data gathered by tracker;
\item Admin marker creation need to be added, also need to add the markers to the page replay and page real-time.
\item CSS and html features need to be updated.
\end{itemize}


